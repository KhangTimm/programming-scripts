\documentclass[twoside]{article}
\usepackage[utf8]{inputenc}
%\usepackage[uebung, answers]{tumbgdm}
\usepackage[uebung]{tumbgdm}
\usepackage{lecturesetupcf2}
\nummerUebungsblatt{3}
\nummerErsteFrage{7}

\LearningOutcome{
  Repeat all information, if you have not been in class, read given book chapters, see
  Moodle page. In the tutorials, make sure everyone knows what is going on.
}

\begin{document}
\maketitle

\begin{task}{Repeat and Simulate}{}{}
  Learn the chapter on digital electronics in detail, use logisim as a simulation tool and
  design your own circuits including simplification and simple state machines.

  For example, you could implement the adder from the last sheet and the most simple, but very important state machine: the counter. This is a circuit which has one input for the clock and outputs the number of clock cycles seen. Implement it with a D flipflip.

  As a challenge, you can implement an SPI engine using shift registers that transmits a byte back and forth given into the shift registers.
\end{task}
\end{document}
