\documentclass[twoside]{article}
\usepackage[utf8]{inputenc}
%\usepackage[uebung, answers]{tumbgdm}
\usepackage[uebung]{tumbgdm}
\usepackage{lecturesetupcf2}
\nummerUebungsblatt{4}
\nummerErsteFrage{8}

\LearningOutcome{
Network Routing Principles
}

\begin{document}
\maketitle

\begin{task}{Network Server and Client}{}{}
  Implement a client and a server in the C programming language using Berkeley sockets. The protocol shall be a simple
  chat system. Each client can connect to the server and send a text command. Implement a few commands like
  ``hello'' letting the server give an answer, ``quit'' should be responsed to with a byebye message and a graceful disconnection from server side (shutdown, close).

  You can extend this to a chat system for multiple clients or a classical text adventure, use a Google search to get inspiration.
\end{task}

\begin{task}{Routing Strategies}{}{}
  Given the network example from the beginning of the routing slides, compute optimal solutions for the named strategies (e.g., minimum maximum energy, etc.).
\end{task}

\begin{task}{Distance Vector Routing}{}{}
  On paper, run a few steps of DSDV on the network from slide 22 (before mobility, from zero knowledge). Assume that every node sends all information. As a second example, assume the network information is already fully distributed (every node knows a correct and complete table). For simplicity, assume all sequence numbers are one. Then, consider the mobility scenario from the slides and run a few updates (on paper). Assume that only changed lines are exchanged with neighbors.
\end{task}
\end{document}
