\begin{task}{Method of Least Squares\up*}{}{}
When working with measurements from the real world errors are
unavoidable. These errors can often be reduced by using multiple
measurements. Since these measurements will contradict each other we
need a method to resolve this contradiction.

Assuming, that the relationship between measurements and the actual
values, we can create the system of overdetermined linear equation

\[
Ax = b
\] where \(b = (b_1,\ldots,b_n)\) contains the measurements, \(x\)
the actual value we want to infer and \(A\) expresses the
\emph{theoretic} relationship between them. We also assume that \(A\)
has maximal rank, meaning that the column vectors of \(A\) are linearly
independant. Because the perfect solution for \(x\) does not exist, due
to the errors in measurement, we are looking for the value of x which
minimizes the \emph{norm} of the residium \(r(x)\):

\[
\sqrt{\sum_{i=1..n}{r_i^2}} = ||\begin{pmatrix}r_1\\...\\r_i\\...\\r_n\end{pmatrix}||=||r(x)|| = ||b-Ax||\]

A perfect solution would be a a \emph{norm} of zero. For the Euclidian
norm we can simplify the term by squaring both sides, as the norm is
defined as the square root of the scalar product, therefore a minimal
norm corresponds to a minimal scalar product. We can formulate the
following equation:

\[ || r(x) || ^2 =(b-Ax)^T(b-Ax) = x^T A^T Ax - 2x^T A^T b + b^T b \rightarrow min\]

As we want to find the minimum we can use differential calculus and set
the first derivative to zero:

\[\frac{\partial}{\partial x} ||r(x)||^2 = 2A^T Ax - 2 A^T b = 0\]

This equation is called \emph{normal equation of an overdetermined
  system of linear equation} \(Ax=b\) and is usually given in the
equivalent form:

\[A^T Ax = A^T b\]

\(A^T A\) is a positive semi-definite, \textbf{symmetric} matrix. As a
result the equation can easily be solved for \(x\) as it is not
overdetermined anymore. This \(x\) minimizes the error function norm.
The Gauss-Markov theorem actually proves that this algorithm provides a
best, linear, unbiased estimate value of \(x\). Which is in asense the
most probable value of \(x\) given the set of measurements.

\begin{enumerate}
\item{
  Using the measured values \code{b = [14.1000;25.9000;18.0500;41.9500]} and a matrix of
  \code{A = [1,2,3;4,5,4;6,3,2;7,7,7]}, calculate the normal equation in MATLAB. Note that matrix transposition is available
  in MATLAB either as a symbol like in \code{A'} or as a function like \code{transpose(A)}.
}
\item{
Solve the normal equation set up in the previous task with MATLAB. Note that it is not good practice to compute the inverse of the matrix $A$, as its realization in main memory can produce fatal rounding errors. It is better to recall the right division in MATLAB from a previous sheet.
  }
\item {
  As the equation is overdetermined, the solution will not be exact. Compute the vector of residual values
  \[
     r = b-Ax
     \]
     and its size
     \[
     ||r||
     \]
     For the size, consider using the MATLAB function \code{norm}.
  }

    
  \end{enumerate}
\end{task}
