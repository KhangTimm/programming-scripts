\begin{task}{Boolean Expressions}{}{}

  The Boolean Algebra is an interesting algebra as it allows to be implemented quite easily in digital electronic circuits.
  In this task, we will try to implement and simplify some specific digital circuits.
    
  \begin{enumerate}
  \item{\textbf{NAND is a universal gate:} From the lecture, we know that AND and NOT suffice to implement a min-term and that OR suffices to implement every binary function $\mathbb{B}^k \rightarrow \mathbb{B}$. Implement these three gates in terms of a NAND gate only.}
  \item{\textbf{Truth Table and Expressions:} Build a Truth Table of 3 input variables and assign an output with 5 ones (randomly placed as you like). Derive the SOP representation, simplify it using algebraic identities, and implement the shortest circuit you can find.}
  \item{\textbf{Combinatorial Complexity:} How many Boolean functions $\mathbb{B}^k \rightarrow \mathbb{B}^l$ do exist?
    How does this compare to the number of atoms in the universe? This task shows that the combinatorial complexity
    is so high that verification of circuits is going to be a challenge. }
  \end{enumerate}
%
\end{task}
