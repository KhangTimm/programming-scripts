\begin{task}{Time Discrete Simulation of Movement According to Basic Physics\up*}{}{}

  Simulations are one of the many kinds of programs we can apply in engineering.
  Simple physics simulations run as so-called \textbf{time-discrete simulations} in a loop that advances a
  shared variable $t$  holding the time by a a discrete timestep $\Delta t$ in every iteration.

  That is, the basic pattern of time-discrete simulation is as follows:
  \begin{lstlisting}
    t_start = 1  % some sort of start time
    t_end = 100  % when to end
    for time = t_start:t_end
    % proceed for a timestep with simulation logic
    % proceed for a timestep with physics
    end
  \end{lstlisting}
  We want to practice this principle.

  \begin{enumerate}
  \item{\textbf{Throwing a Ball:} When throwing a ball, you use your body to assign a certain velocity (e.g., speed acting into a known direction) to it. How does this given speed combined with gravity complete to a time-discrete simulation returing the distance of your pitch? Complete the following simulation.
    \begin{lstlisting}[language=Matlab]
p = [0,1] % initial position
angle = 45 % start angle
speed = 10 % inital speed in m/s
v = speed*[cos(deg2rad(angle)),sin(deg2rad(angle))]  % velocity
g = [0,-9.81] % gravity
dt = 0.2 % delta time
dg = g*dt % delta gravity
L = [p] % Lines

for i = 1:100
                    % <-- update speed
                    % <-- update location
                    % <-- break when hitting floor


                    % <-- add  point to trajectory
    disp(p(2));
end

disp(sprintf("we went for %f m in %i iterations", L(end,1), i))

% visualize
plot(L(:,1),L(:,2))

      \end{lstlisting}

    
  }
  \item{\textbf{Optimal Angle:} Adjust the program in such a way, that you find the best integer angle, that achieves the largest distance and have the program display the best angle as well as plot the best trajectory. Tipp: As we constrained to integer angles, you can use a keep-the-best pattern similar to the following:
    \begin{lstlisting}[language=Matlab]
best_angle = -1; % We dont know our best angle
best_length = -inf; % We dont know the best length
for this_angle = 1:360
    this_length = simulate_throw(this_angle)
    if this_length > best_length
        fprintf("Updating best_length from %.2f to %.2f at angle %d\n", best_length, this_length, this_angle)
        best_length = this_length
        best_angle = this_angle
    end
end
fprintf("The best result of %.2f m was achieved with an angle of %d\n",best_length, best_angle)

    \end{lstlisting}
    In this pattern, \code{-inf} denotes minus infinity and every finite number is larger. We then run the loop and if there is an improvement, we update the variables \code{best\_angle} and \code{best\_length}. 

  }
  \end{enumerate}
  

  
\end{task}

%
%
